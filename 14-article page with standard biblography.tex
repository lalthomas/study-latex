%% Thanks https://www.sharelatex.com/learn/Bibliography_management_with_bibtex

\documentclass{article}
\begin{document}
\section{Section}
Lorem ipsum dolor sit amet, consectetur adipisicing elit. Eius, cum, explicabo, dicta, natus animi aperiam incidunt doloremque inventore illo recusandae tempora beatae ipsum dolorum molestiae sapiente porro quos alias ex!
\\ Lorem ipsum dolor sit amet, consectetur adipisicing elit. Quos, veniam, commodi, incidunt optio quam maxime vero tempore accusantium quasi corrupti impedit corporis quod sint quae excepturi quia laudantium esse vel.
\\ Lorem ipsum dolor sit amet, consectetur adipisicing elit. Eius, cum, explicabo, dicta, natus animi aperiam incidunt doloremque inventore illo recusandae tempora beatae ipsum dolorum molestiae sapiente porro quos alias ex!
\\ Lorem ipsum dolor sit amet, consectetur adipisicing elit. Quos, veniam, commodi, incidunt optio quam maxime vero tempore accusantium quasi corrupti impedit corporis quod sint quae excepturi quia laudantium esse vel.

Latex Companion \cite{latexcompanion} Einstein \cite{einstein}

%% Standard bibliography commands in LaTeX have a similar syntax to that of lists and items. 
\begin{thebibliography}{9}
%% textit - make text italic
\bibitem{latexcompanion} Michel Goossens, Frank Mittelbach, and Alexander Samarin. \textit{The \LaTeX\ Companion}. Addison-Wesley, Reading, Massachusetts, 1993.
\bibitem{einstein} Albert Einstein. \textit{Zur Elektrodynamik bewegter K{\"o}rper}. (German) [\textit{On the electrodynamics of moving bodies}]. Annalen der Physik, 322(10):891-921, 1905.
%% texttt - typewriter font
\bibitem{knuthwebsite} Knuth: Computers and Typesetting, \\\texttt{http://www-cs-faculty.stanford.edu/\~{}uno/abcde.html}
\end{thebibliography}
\end{document}
 

 